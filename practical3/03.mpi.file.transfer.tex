
```latex
\documentclass{article}
\usepackage[utf8]{inputenc}
\usepackage{listings}
\usepackage{graphicx}
\usepackage{hyperref}

\title{Practical Work 3: MPI File Transfer}

\begin{document}

\maketitle

\section{MPI Implementation Choice}
For this project, I chose \textbf{OpenMPI} combined with the \textbf{mpi4py} Python library. This choice was made because OpenMPI is an open-source, high-performance implementation that is widely supported across different platforms. Using \texttt{mpi4py} allows for rapid development while maintaining the efficiency of MPI's message-passing capabilities in a distributed environment.

\section{Design and Organization}
The system is designed using a **Coordinator-Worker** pattern.

\subsection{System Organization}
The organization consists of one central node and multiple processing nodes:
\begin{itemize}
    \item \textbf{Coordinator (Rank 0):} Acts as the server/source that holds the original file.
    \item \textbf{Workers (Rank 1 to $N-1$):} Act as the clients/receivers that store the transferred data.
\end{itemize}

\subsection{Service Design}
The file transfer process follows these steps:
\begin{enumerate}
    \item The Coordinator reads the file metadata and sends it to all workers using Tag 1.
    \item The file is split into chunks of $4096$ bytes.
    \item Chunks are distributed to workers using Tag 2 in a round-robin sequence.
    \item Once the file is fully read, the Coordinator sends an end signal using Tag 3 to stop worker processes.
\end{enumerate}

\section{Implementation}
The core logic for distributing file chunks is implemented as follows:

\begin{verbatim}
# Coordinator chunk distribution logic
with open(filename, 'rb') as f:
    while True:
        chunk_data = f.read(CHUNK_SIZE)
        if not chunk_data:
            break
        
        chunk = {'id': chunk_id, 'data': chunk_data}
        MPI.COMM_WORLD.send(chunk, dest=current_worker, tag=2)
        
        chunk_id += 1
        current_worker = (current_worker % num_workers) + 1
\end{verbatim}


\section{Roles and Responsibilities}
\begin{table}[h]
\centering
\begin{tabular}{|l|l|}
\hline
\textbf{Role} & \textbf{Responsibility} \\ \hline
Coordinator (Rank 0) & File I/O, Chunking, Load Balancing (Round-robin) \\ \hline
Workers (Rank 1+) & Data reception, Writing to local disk, Progress reporting \\ \hline
\end{tabular}
\caption{Who does what in the system}
\end{table}

\end{document}